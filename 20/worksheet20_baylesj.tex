\documentclass[12pt,letterpaper]{article}

\author{Jordan Bayles}
%\date{}
\title{}

%Usepackage declarations
\usepackage[left=1in,top=.5in,right=1in,bottom=.5in]{geometry}
%\usepackage[T1]{fontenc}
%\usepackage{tgpagella}
%\usepackage[protrusion=true,expansion=true]{microtype}
\usepackage{xcolor}
\usepackage{color}
\usepackage{fancyhdr}
\usepackage{lastpage}
\usepackage{sectsty}
\usepackage{slashed}
\usepackage{amsmath}
\usepackage{amsfonts}
\usepackage{listings}
\usepackage{latexsym}
% Include for easy import of full pdf pages
\usepackage{pdfpages}
% Include for use of images
\usepackage{graphicx}
% Include for use of [H] placement specifier
\usepackage{float}
% Include for use of \toprule, \midrule, \bottomrule in tabular env.
\usepackage{booktabs}
% Include for setting spacing between lines
\usepackage{setspace}

%Package usages
\sectionfont{\normalsize}
\subsectionfont{\small}

%%Fancy Header setup
\pagestyle{fancy}
% Clear default
\fancyhead{}
\fancyfoot{}
%New settings
\fancyhead[LE,RO]{\slshape \rightmark}
\fancyhead[LO,RE]{\slshape \leftmark}
\fancyfoot[C]{\thepage}
\renewcommand{\headrulewidth}{0.4pt}
\renewcommand{\footrulewidth}{0.4pt}

%New commands
\newcommand{\comment}[1]{}
\newcommand{\field}[1]{\mathbb{#1}} % requires amsfonts
\newcommand{\script}[1]{\mathcal{#1}} % requires amsfonts
\newcommand{\pd}[2]{\frac{\partial#1}{\partial#2}}

\usepackage{listings}
\usepackage{color}
\usepackage[font=small,format=plain,labelfont=bf,up,textfont=it,up]{caption}
 
\definecolor{dkgreen}{rgb}{0,0.6,0}
\definecolor{gray}{rgb}{0.5,0.5,0.5}
\definecolor{mauve}{rgb}{0.58,0,0.82}
\definecolor{lightgrey}{gray}{0.8}
\definecolor{darkgrey}{gray}{1.6}

\DeclareCaptionFormat{listing}{\colorbox{gray}{\parbox{0.987\linewidth}{#1#2#3}}}
\captionsetup[lstlisting]{format=listing, labelfont=white, indention=0pt, textfont=white, margin=0pt, font={bf,footnotesize}, singlelinecheck=false}
\DeclareCaptionFont{white}{\color{white}}

\renewcommand{\lstlistingname}{Code}
\lstset{ %
  %Some lang opts: C++, C, Java, Python, Matlab, TeX, HTML, PHP, SQL, Verilog, VHDL, make, ...
  language=Octave,                    % the language of the code
  basicstyle=\footnotesize\ttfamily , % the size of the fonts that are used for the code
  numbers=left,                       % where to put the line-numbers
  numberstyle=\scriptsize\color{darkgray}, % the style that is used for the line-numbers
  stepnumber=2,                       % the step between two line-numbers. 
  numbersep=5pt,                      % how far the line-numbers are from the code
  backgroundcolor=\color{white},      % choose the background color. You must add \usepackage{color}
  showspaces=false,                   % show spaces adding particular underscores
  showstringspaces=false,             % underline spaces within strings
  showtabs=false,                     % show tabs within strings adding particular underscores
  frame=tb,                           % adds a frame around the code
  rulesepcolor=\color{gray},          % if not set, the frame-color may be changed on line-breaks within not-black text (e.g. commens (green here))
  tabsize=2,                          % sets default tabsize to 2 spaces
  captionpos=t,                       % sets the caption-position
  breaklines=true,                    % sets automatic line breaking
  breakatwhitespace=false,            % sets if automatic breaks should only happen at whitespace
  title=\lstname,                     % show the filename of files included with \lstinputlisting;
  keywordstyle=\color{blue},          % keyword style
  commentstyle=\color{dkgreen},       % comment style
  stringstyle=\color{mauve},          % string literal style
  escapeinside={\%*}{*)},             % if you want to add a comment within your code
  morekeywords={*,...}                % if you want to add more keywords to the set
  framexbottommargin=5pt,
}
\begin{document}
%\maketitle

\begin{flushright}
Jordan Bayles\\
Worksheet 20\\
\end{flushright}
\begin{center}
Worksheet 20 Implementation
\end{center}

%Sample Code listing
\lstset{caption={Instructor provided code},label=DescriptiveLabel,language=C}
\begin{lstlisting}
void _dequeSetCapacity (struct deque *d, int newCap) {
	int i;
	/* Create a new underlying array*/
	TYPE *newData = (TYPE*)malloc(sizeof(TYPE)*newCap);
	assert(newData != 0);
	/* copy elements to it */
	int j = v->beg;
	for(i = 0; i <  v->size; i++)
	{
		newData[i] = v->data[j];
		j = j + 1;
		if(j >= v->capacity)
			j = 0;
	}
	/* Delete the oldunderlying array*/
	free(v->data);
	/* update capacity and size and data*/
	v->data = newData;
	v->capacity = newCap;
	v->beg = 0;
}

void dequeFree (struct deque *d) {
   free(d->data); d->size = 0; d->capacity = 0;
}

struct deque {
	TYPE * data;
	int capacity;
	int size;
	int start;
};

void dequeInit (struct deque *d, int initCapacity) {
   d->size = d->beg = 0;
   d->capacity = initCapacity; assert(initCapacity > 0);
   d->data = (TYPE *) malloc(initCapacity * sizeof(TYPE));
   assert(d->data != 0);

}
int dequeSize (struct deque *d) { return d->size; }
void _dequeDoubleCapacity (struct deque *d);  
\end{lstlisting}
\newpage
\lstset{caption={Add to front function},label=DescriptiveLabel,language=C}
\begin{lstlisting}
void dequeAddFront (struct deque *d, TYPE newValue) {
	if (d->size >= d->capacity) _dequeSetCapacity(d, 2*d->capacity);
	
	d->start -= 1;	// Shift starting position back to add new element
	d->size += 1;
	d->data[ d->start ] = newValue;	// Insert element at new starting position
	
}
\end{lstlisting}

\lstset{caption={Add to back function},label=DescriptiveLabel,language=C}
\begin{lstlisting}
void dequeAddBack (struct deque *d, TYPE newValue) {
	int end;
    if (d->size >= d->capacity) _dequeSetcapacity(d, 2* d->capacity);
    
    // d->start + d->size should point to the spot for the *next* element at back
    end = d->start + d->size;
    end -= ( end >= d->capacity ) ? d->capacity : 0;
    
    d->data[ end ] = newValue;
    d->size += 1;
}
\end{lstlisting}

\lstset{caption={Return front function},label=DescriptiveLabel,language=C}
\begin{lstlisting}
TYPE dequeFront (struct deque *d) {
	return( d->data[ d->start ] );
}
\end{lstlisting}

\lstset{caption={Return back function},label=DescriptiveLabel,language=C}
\begin{lstlisting}
TYPE dequeBack (struct deque *d) {
    int end = d->start + d->size;
    end -= ( end >= d->capacity ) ? d->capacity : 0;
    return( d->data[ end ] );
}
\end{lstlisting}

\lstset{caption={Remove front function},label=DescriptiveLabel,language=C}
\begin{lstlisting}
void dequeRemoveFront (struct deque *d) {
	d->data[ start ] = 0;
	d->start += 1;
	d->size -= 1;
}
\end{lstlisting}

\lstset{caption={Remove back function},label=DescriptiveLabel,language=C}
\begin{lstlisting}
void dequeRemoveBack (struct deque *d) {
	int end = d->start + d->size;
    end -= ( end >= d->capacity ) ? d->capacity : 0;
    d->data[ end ] = 0; // Enough to decrement size, but explicit better than implicit
    d->size -= 1;
}
\end{lstlisting}

\end{document}