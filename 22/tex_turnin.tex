\documentclass[10pt,letterpaper]{article}

\author{Jordan Bayles\\
        Cameron Evensen\\
        Marshal Horn\\
        Robert Plascencia}
%\date{}
\title{JCMR, Worksheet 22}

%Usepackage declarations
\usepackage[left=1in,top=1in,right=1in,bottom=1in]{geometry}
\usepackage[T1]{fontenc}
\usepackage{tgpagella}
\usepackage[protrusion=true,expansion=true]{microtype}
\usepackage{xcolor}
\usepackage{color}
\usepackage{fancyhdr}
\usepackage{lastpage}
\usepackage{sectsty}
\usepackage{slashed}
\usepackage{amsmath}
\usepackage{amsfonts}
\usepackage{listings}
\usepackage{latexsym}
% Include for easy import of full pdf pages
\usepackage{pdfpages}
% Include for use of images
\usepackage{graphicx}
% Include for use of [H] placement specifier
\usepackage{float}
% Include for use of \toprule, \midrule, \bottomrule in tabular env.
\usepackage{booktabs}
% Include for setting spacing between lines
\usepackage{setspace}

%Package usages
\sectionfont{\normalsize}
\subsectionfont{\small}

%%Fancy Header setup
\pagestyle{fancy}
% Clear default
\fancyhead{}
\fancyfoot{}
%New settings
\fancyhead[LE,RO]{\slshape \rightmark}
\fancyhead[LO,RE]{\slshape \leftmark}
\fancyfoot[C]{\thepage}
\renewcommand{\headrulewidth}{0.4pt}
\renewcommand{\footrulewidth}{0.4pt}

%New commands
\newcommand{\comment}[1]{}
\newcommand{\field}[1]{\mathbb{#1}} % requires amsfonts
\newcommand{\script}[1]{\mathcal{#1}} % requires amsfonts
\newcommand{\pd}[2]{\frac{\partial#1}{\partial#2}}

\usepackage{listings}
\usepackage{color}
\usepackage[font=small,format=plain,labelfont=bf,up,textfont=it,up]{caption}
 
\definecolor{dkgreen}{rgb}{0,0.6,0}
\definecolor{gray}{rgb}{0.5,0.5,0.5}
\definecolor{mauve}{rgb}{0.58,0,0.82}
\definecolor{lightgrey}{gray}{0.8}
\definecolor{darkgrey}{gray}{1.6}

\DeclareCaptionFormat{listing}{\colorbox{gray}{\parbox{0.987\linewidth}{#1#2#3}}}
\captionsetup[lstlisting]{format=listing, labelfont=white, indention=0pt, textfont=white, margin=0pt, font={bf,footnotesize}, singlelinecheck=false}
\DeclareCaptionFont{white}{\color{white}}

\renewcommand{\lstlistingname}{Code}
\lstset{ %
  %Some lang opts: C++, C, Java, Python, Matlab, TeX, HTML, PHP, SQL, Verilog, VHDL, make, ...
  language=Octave,                    % the language of the code
  basicstyle=\small\ttfamily,         % the size of the fonts that are used for the code
  numbers=left,                       % where to put the line-numbers
  numberstyle=\scriptsize\color{darkgray}, % the style that is used for the line-numbers
  stepnumber=2,                       % the step between two line-numbers. 
  numbersep=5pt,                      % how far the line-numbers are from the code
  backgroundcolor=\color{white},      % choose the background color. You must add \usepackage{color}
  showspaces=false,                   % show spaces adding particular underscores
  showstringspaces=false,             % underline spaces within strings
  showtabs=false,                     % show tabs within strings adding particular underscores
  frame=tb,                           % adds a frame around the code
  rulesepcolor=\color{gray},          % if not set, the frame-color may be changed on line-breaks within not-black text (e.g. commens (green here))
  tabsize=2,                          % sets default tabsize to 2 spaces
  captionpos=t,                       % sets the caption-position
  breaklines=true,                    % sets automatic line breaking
  breakatwhitespace=false,            % sets if automatic breaks should only happen at whitespace
  title=\lstname,                     % show the filename of files included with \lstinputlisting;
  keywordstyle=\color{blue},          % keyword style
  commentstyle=\color{dkgreen},       % comment style
  stringstyle=\color{mauve},          % string literal style
  escapeinside={\%*}{*)},             % if you want to add a comment within your code
  morekeywords={*,...}                % if you want to add more keywords to the set
  framexbottommargin=5pt,
}
\begin{document}
\begin{flushright}
Jordan Bayles\\
Cameron Evensen\\
Marshal Horn\\
Robert Plascencia
\end{flushright}

\begin{center}
JCMR, Worksheet 22
\end{center}

\section{Code implementation}
%Sample Code listing
\lstset{caption={Worksheet \#22 - Code implementation},label=DescriptiveLabel,language=C}
\lstinputlisting[language=C]{code_final.c}
\newpage

\section{Question responses}
\subsection
{
What were the algorithmic complexities of the methods addLink and removeLink
that you wrote back in Chapter Q?
}
Given that the search phrase ``Chapter Q'' gives exactly one result for the class
documentation (this worksheet), I will assume that is some sort of type or
stale reference. However, based upon the methods written in this worksheet
the answer is much clearer. The \verb!_addLink! function accepts a pointer
to the list and the link to add before, same for the \verb!_removeLink! function.

\subsection
{
Given your answer to the previous question, what are the algorithmic complexities
of the three principle Bag operations?
}
Because the \verb!_addLink! and \verb!_removeLink! functions are constant time
(the nearby link is passed in), the algorithmic complexities of the add,
remove and contains Bag operations can be found in their respective linkedList*
functions:

\begin{itemize}
    \item \verb!linkedListAdd! is constant time (front of linkedList), or
        $\mathcal{O}( 1 )$.
    \item \verb!linkedListContains! is linear time, as it must iterate through
        the linked list in order to find the element to return its value. Worst
        case is $n$ operations, so the contains operation is $\mathcal{O}(n)$.
    \item Although the actual remove step is constant time, because \verb!linkedListRemove!
        must first iterate through the list to find the element to remove (similar
        to the contains function), \verb!linkedListRemove! is also $\mathcal{O}(n)$.
\end{itemize}

\end{document}
